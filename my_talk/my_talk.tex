\documentclass{beamer}

\mode<presentation>
{
  \usetheme{Frankfurt}
  \usecolortheme{seahorse}
  \setbeamercovered{transparent}
  \usefonttheme{structurebold}
}

\usepackage[english]{babel}
\usepackage[latin1]{inputenc}
\usepackage{times}
\usepackage[T1]{fontenc} 
% Or whatever. Note that the encoding and the font should match. If T1
% does not look nice, try deleting the line with the fontenc.
\usepackage{amsmath}

\newcommand{\linespace}{\vskip 0.25cm}

\definecolor{MyForestGreen}{rgb}{0,0.7,0} 
\newcommand{\tableemph}[1]{{#1}}
\newcommand{\tablewin}[1]{\tableemph{#1}}
\newcommand{\tablemid}[1]{\tableemph{#1}}
\newcommand{\tablelose}[1]{\tableemph{#1}}

\definecolor{MyLightGray}{rgb}{0.6,0.6,0.6}
\newcommand{\tabletie}[1]{\color{MyLightGray} {#1}}

% The text in square brackets is the short version of your title and will be used in the
% header/footer depending on your theme.
\title[AAD through GPHH]{Automating Algorithm Design through\\Genetic Programming Hyper-Heuristics}

% Sub-titles are optional - uncomment and edit the next line if you want one.
% \subtitle{Why does sub-tree crossover work?} 

% The text in square brackets is the short version of your name(s) and will be used in the
% header/footer depending on your theme.
\author[Browning]{Elsa Browning}

% The text in square brackets is the short version of your institution and will be used in the
% header/footer depending on your theme.
\institute[U of Minn, Morris]
{
  Division of Science and Mathematics \\
  University of Minnesota, Morris \\
  Morris, Minnesota, USA
}

% The text in square brackets is the short version of the date if you need that.
\date[April '17] % (optional)
{April 15, 2017\\ Morris, MN}

% Delete this, if you do not want the table of contents to pop up at
% the beginning of each subsection:
\AtBeginSection[]
{
  \begin{frame}<beamer>
    \frametitle{Outline}
    \tableofcontents[currentsection, hideothersubsections]
  \end{frame}
}

\begin{document}

\begin{frame}
  \titlepage
\end{frame}

% For a 20-25 minute senior seminar talk you probably want something like:
% - Two or three major sections (other than the summary).
% - At *most* three subsections per section.
% - Talk about 30s to 2min per frame. So there should probably be between
%   15 and 30 frames, all told.

\section*{Overview}

\subsection*{What does the title mean?}

\begin{frame}
	\frametitle{What does the title mean?}
	\begin{columns}
		\begin{column}{0.7\textwidth}
			\begin{itemize}
				\item Taking the human element out of algorithm design
				\linespace
				\item More work at the beginning, more possibilities
				\linespace
				\item Genetic programming hyper-heuristics as a method to the madness
			\end{itemize}
		\end{column}
		\begin{column}{0.3\textwidth}
			[Insert cool relevant picture(s)]
		\end{column}
	\end{columns}
\end{frame}

\subsection*{Outline}

\begin{frame}
	\frametitle{Outline}
	\tableofcontents[hideallsubsections]
\end{frame}

\section[Background]{Background}

\subsection{Evolutionary Computation}

%\begin{frame}
%	\frametitle{Evolutionary Computation}
%\end{frame}

\subsection{Genetic Programming}

%\begin{frame}
%	\frametitle{Genetic Programming}
%\end{frame}

\section[Hyper-heuristics]{Hyper-heuristics}

\subsection{What they are}
\subsection{What they aren't}
\subsection{How they work}

%\begin{frame}
%	\frametitle{What are hyper-heuristics?}
%\end{frame}

\section[GP Variants]{Genetic Programming Variants}
\subsection{Why they matter}
\subsection{Stack-based genetic programming}

\section[Autoconstruction]{Autoconstruction}

\subsection{What is it?}

%\begin{frame}
%	\frametitle{What is Autoconstruction?}
%\end{frame}

\subsection{AutoDoG}

%\begin{frame}
%	\frametitle{The system called AutoDoG}
%\end{frame}

\subsection{Push language}
\subsection{Lexicase selection}
\subsection{Results}

\section[Summary]{Summary}

\end{document}